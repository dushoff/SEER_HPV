
\usepackage{amsmath} % essential for cases environment
\usepackage{amsthm} % for theorems and proofs
\usepackage{amsfonts} % mathbb
\usepackage{marvosym}
\usepackage{xspace}
\usepackage{multirow} % fancy tables
\usepackage{wasysym} % circle symbols (including half-filled circles)
\usepackage{enumerate} % fancier enumeration (e.g., a,b,c, ...)
%\usepackage{xcolor}
\usepackage{color}
\usepackage{mathtools}
\usepackage{tikz}
\usepackage{oubraces}
\usepackage{hyperref}
\usepackage[toc,page]{appendix}
\usepackage{cleveref}
\usepackage{cite}
\usepackage{textcomp}

%<<<<<<< HEAD
%=======
% \usepackage{algpseudocode,algorithm}

\usepackage{caption}
%used for spacing in list of figures/tables
\usepackage{tocloft}
%predominantly used for list of abbreviations and symbols
\usepackage{longtable}
\usepackage{enumitem}
	\newlist{abbrv}{itemize}{1}
	\setlist[abbrv,1]{label=,labelwidth=1in,align=parleft,itemsep=0.1\baselineskip,leftmargin=!}

%langauge
\usepackage[english]{babel}

%Tabular Commands
\usepackage{array}
\newcolumntype{L}[1]{>{\raggedright\let\newline\\\arraybackslash\hspace{0pt}}m{#1}}
\newcolumntype{C}[1]{>{\centering\let\newline\\\arraybackslash\hspace{0pt}}m{#1}}
\newcolumntype{R}[1]{>{\raggedleft\let\newline\\\arraybackslash\hspace{0pt}}m{#1}}

%make align work with lineno

%\newcommand*\patchAmsMathEnvironmentForLineno[1]{%
%  \expandafter\let\csname old#1\expandafter\endcsname\csname #1\endcsname
%  \expandafter\let\csname oldend#1\expandafter\endcsname\csname end#1\endcsname
%  \renewenvironment{#1}%
%     {\linenomath\csname old#1\endcsname}%
%     {\csname oldend#1\endcsname\endlinenomath}}% 
%\newcommand*\patchBothAmsMathEnvironmentsForLineno[1]{%
%  \patchAmsMathEnvironmentForLineno{#1}%
%  \patchAmsMathEnvironmentForLineno{#1*}}%
%\AtBeginDocument{%
%\patchBothAmsMathEnvironmentsForLineno{equation}%
%\patchBothAmsMathEnvironmentsForLineno{align}%
%\patchBothAmsMathEnvironmentsForLineno{flalign}%
%\patchBothAmsMathEnvironmentsForLineno{alignat}%
%\patchBothAmsMathEnvironmentsForLineno{gather}%
%\patchBothAmsMathEnvironmentsForLineno{multline}%
%}

%commenting commands
\newcommand{\spenny}[1]{{\color{red}{(\bfseries Spenny: }{\em #1})}}
%\renewcommand{\spenny}[1]{}

%colours

\definecolor{dodgerblue}{RGB}{30,144,255}
\definecolor{darkorchid1}{RGB}{172,29,255}
\definecolor{orange}{RGB}{255,149,0}
\definecolor{forestgreen}{RGB}{0,122,16}
\newcommand{\red}[1]{{\color{red}#1}}

\newtheorem{theorem}{Theorem}[section]
\newtheorem{lemma}{Lemma}[section]
%\renewcommand\qedsymbol{\Coffeecup}
\newcommand{\Note}[1]{\textbf{\emph{Note:}\xspace#1}}
%Greek Letter shortcuts
\newcommand{\lam}{\lambda}
\newcommand{\Lam}{\Lambda}
\newcommand{\gam}{\gamma}
\newcommand{\Gam}{\Gamma}
\newcommand{\eps}{\varepsilon}


\newcommand{\ee}{(\hat{P_1},\hat{P_2},\hat{P_{12}})}
\newcommand{\eep}{\left(1-\frac{\mu}{f_1}\right)}
\newcommand{\eef}{\left(1-\frac{\mu}{f_1},0,0\right)}
\newcommand{\JD}[1]{{\color{blue}{\bfseries Jonathan:} #1}}
\newcommand{\EEZone}{\frac{\beta}{\alpha}\left(1-\frac{\mu}{\gam}\right)}

%Simulation Commands/Macros
\newcommand{\neigh}{{\cal N}}
\newcommand{\state}{\text{state}}
\newcommand{\cmax}[1]{\ensuremath c_{\rm #1}}
\newcommand{\heal}{\text{H}}
\newcommand{\susc}{\text{S}}
\newcommand{\expose}{\text{E}}
\newcommand{\infect}{\text{I}}


%Commenting commands:
\newcommand{\Question}[1]{{\em \bf Question:} #1}
\newcommand{\Answer}[1]{{\em \bf Answer:} #1}

%Unit commands
\newcommand{\mum}{\ensuremath \mu{\rm m}}
\newcommand{\cm}{\ensuremath {\rm cm}}
\newcommand{\mm}{\ensuremath {\rm mm}}

\newcommand{\f}{f}

%Equilibrium macros
\newcommand{\HE}{\textit{HE}\xspace}
\newcommand{\DE}{\textit{DE}\xspace}
\newcommand{\equil}{(\bar{H},\bar{S},\bar{E},\bar{I},\bar{V},\bar{Z})}
\newcommand{\eq}[1]{\overline{#1}}


%% macros
\newcommand{\Reals}{\mathbb{R}}
\newcommand{\term}[1]{{\bfseries\slshape #1}}
\newcommand{\Ker}{{\text{Ker}\,}}
\newcommand{\argmax}{{\text{argmax}}}
\newcommand{\argmin}{{\text{argmin}}}
\newcommand{\Range}{{\text{Range}\,}}
\newcommand{\norm}[1]{\left\|#1\right\|}
\newcommand{\abs}[1]{\left|#1\right|}
\newcommand{\R}{{\cal R}}
\newcommand{\G}{{\cal G}}
\newcommand{\N}{{\cal N}}
\newcommand{\Tinf}{T_\textrm{inf}}
\newcommand{\Prob}{\textrm{Pr}}
\newcommand{\Shat}{{\hat{S}}}
\newcommand{\Ihat}{{\hat{I}}}
\newcommand{\ie}{\emph{i.e., }}
\newcommand{\eg}{\emph{e.g., }}
\newcommand{\Rlogo}{\textbf{\textsf{R}}\xspace}
\newcommand{\XPPAUT}{\texttt{XPPAUT}\xspace}
\newcommand{\etal}{\textit{et al}.\xspace}
\newcommand\emphblue[1]{\emph{\color{blue}#1}}
\newcommand{\citehere}{{\large \bf CITE HERE}}

%derivative notation
\newcommand{\D}[2]{\frac{\mathrm{d}#1}{\mathrm{d}#2}}
\newcommand{\partD}[2]{\frac{\partial \mathrm{d}#1}{\partial #2 			\mathrm{d}t}}
\newcommand{\at}[2][]{\left. #1\right|_{#2}}
\newcommand{\partd}[2]{\frac{\partial #1}{\partial #2}}
\newcommand{\x}{\text{\bf x}}
\newcommand{\Mod}[1]{\ (\text{mod}\ #1)}
%THESE ARE SPENCER'S MACROSSSSSSS

\newcommand{\A}{\frac{\alpha\delta(\rho+\chi)}{\chi\beta f}}
\newcommand{\perday}{{$\text{day}^{-1}$\xspace}}

%Text Macros
\newcommand{\TM}{\textsuperscript{TM}\xspace}

\definecolor{dkgreen}{rgb}{0,0.6,0}
\definecolor{gray}{rgb}{0.5,0.5,0.5}
\definecolor{mauve}{rgb}{0.58,0,0.82}

%-----------------
% Listings Package for code script
%-----------------
\usepackage{listings}


\lstset{ %
  language=R,                     % the language of the code
  basicstyle=\footnotesize\ttfamily,       % the size of the fonts that are used for the code
  numbers=left,                   % where to put the line-numbers
  numberstyle=\tiny\color{gray},  % the style that is used for the line-numbers
  stepnumber=1,                   % the step between two line-numbers. If it's 1, each line
                                  % will be numbered
  numbersep=5pt,                  % how far the line-numbers are from the code
  backgroundcolor=\color{white},  % choose the background color. You must add \usepackage{color}
  showspaces=false,               % show spaces adding particular underscores
  showstringspaces=false,         % underline spaces within strings
  showtabs=false,                 % show tabs within strings adding particular underscores
  frame=none,                   % adds a frame around the code
  rulecolor=\color{black},        % if not set, the frame-color may be changed on line-breaks within not-black text (e.g. commens (green here))
  tabsize=2,                      % sets default tabsize to 2 spaces
  captionpos=b,                   % sets the caption-position to bottom
  breaklines=true,                % sets automatic line breaking
  breakatwhitespace=false,        % sets if automatic breaks should only happen at whitespace
  title=\lstname,                 % show the filename of files included with \lstinputlisting;
                                  % also try caption instead of title
  keywordstyle=\color{blue},      % keyword style
  commentstyle=\color{dkgreen},   % comment style
  stringstyle=\color{mauve},      % string literal style
  escapeinside={\%*}{*)},         % if you want to add a comment within your code
  morekeywords={*,...}            % if you want to add more keywords to the set
} 
