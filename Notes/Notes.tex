\documentclass[12pt]{article}

\usepackage{amsmath,amsthm,xspace,multirow}
\usepackage[margin=1in]{geometry}
\input{preamble.tex}
\input{sirmodellingmacros.tex}

%\newcommand{\D}[2]{\frac{\mathrm{d}#1}{\mathrm{d}#2}}
%\newcommand{\ie}{\emph{i.e., }}
%\newcommand{\eg}{\emph{e.g., }}

\usepackage{tikz}
\newdimen\mylw
\tikzset{chemeq/.style={draw,thick,double distance=2pt,onearc-onearc,/chemeq/size={#1}}}
\tikzset{chemeq/.default={.4pt 6pt}}
\pgfkeys{/chemeq/size/.code={\pgfsetarrowoptions{onearc}{#1}}}
\def\parseopts#1 #2{\xdef\myalw{#1}\xdef\myasize{#2}}
\pgfarrowsdeclare{onearc}{onearc}{%
  {\edef\x{\pgfgetarrowoptions{onearc}}\expandafter\parseopts\x}
  \mylw=\myalw
  \pgfarrowsleftextend{-\myasize-.5\mylw}
  \pgfarrowsrightextend{0pt}
}{%
  \pgfsetdash{}{0pt}
  {\edef\x{\pgfgetarrowoptions{onearc}}\expandafter\parseopts\x}
  \mylw=\pgflinewidth
  \pgfsetlinewidth{\myalw}
  \pgfpathmoveto{\pgfpoint{-\myasize}{-\myasize-.5\mylw}}%
  \pgfpatharc{180}{90}{\myasize}
  \pgfusepathqstroke
}

\title{HPV Vaccination of Boys and Men}

\begin{document}
\maketitle

\section*{Previous Work - Marek's Work}

Marek Smieja and a small research group because examining this problem during a workshop.  They set up a simple HPV transmission model in men and women.  They consider three infection status groups for both men and women: susceptible, infected, and vaccinated.  Individuals enter the model when they become sexually active (they assume age 15) at a rate $b_iN_i$, where $b_i$ is the birthrate for gender $i$ and $N_i$ is the number of individuals of gender $i$.  Individuals are separated into vaccinated or susceptible at a particular vaccination rate $v_i$ for gender $i$.  

Transmission is considered based on sexual interactions between men and women.  They consider same-sex interactions in men and women, but due to the lack of same-sex data, they set these parameters to zero when fitting the data.  The transmission of HPV in women is expressed as follows (analogously for men):
\begin{equation}
S_{W}\left(\beta_{fm}\frac{I_M}{N_M} + \beta_{ff}\frac{I_W}{N_W}\right)
\end{equation}
where $\beta_{fm}$ and $\beta_{ff}$ is the transmission of HPV onto women from men and women, respectively.  When an individuals is infected, they can clear the infection at a rate $\gamma_i$ after which the individual becomes susceptible again.  There is evidence that natural infections do not provide much protection from subsequent re-infection and does not protect significantly from infections of other HPV types.  

They also consider the vaccination isn't 100\% effective or long lasting. Vaccinated individuals' protection wanes at a rate of $w_i$ for gender $i$, after which they become fully susceptible.  Vaccinated individuals also have a protective factor of $\eps\in[0,1]$, which scales HPV transmission.  That is $\eps=0$ is full protection and $\eps=1$ is no protection.  

We adapt this model by separating the male group into two different subgroups: straight males $M$ and men who have sex with men (queer men) $Q$. In this way we can examine the effects of vaccination strategies on men and queer men.  There is a concern that the female-only vaccination strategies provide little to no herd immunity to MSM and thus they may become a reservoir for HPV. 

\section*{Adaptations to the Model}

As previously mentioned we separate the men into straight men $M$ and queer men $Q$.  Queer men have sexual interactions with both men and and women.  The proportion of same-sex interactions in men is $p$ and the proportion of same-sex interactions in women is $q$.  In this way we set up three different HPV transmission terms:

We examine the transmission of HPV in straight males.  These individuals can be infected by women solely.  
\begin{equation}
S_M\left(\frac{(1-q)\beta_{mf}I_W}{N_W}\right)
\end{equation}
Queer men can be infected by both other MSM and also women.
\begin{equation}
S_Q\left(\frac{p^2\beta_{mm}I_Q}{N_Q}+\frac{(1-p)(1-q)\beta_{mf}I_W}{N_W}\right)
\end{equation}
Women can be infected by both straight and queer men, and we also consider same sex interactions in women. 
\begin{equation}
S_W\left(\frac{(1-q)\beta_{fm}I_M}{N_M}+\frac{(1-q)(1-p)\beta_{fm}I_Q}{N_Q}+\frac{q^2\beta_{ff}I_W}{N_W}\right)
\end{equation}
We set up the entire model as a flow diagram.  The flow diagram for straight men is illustrated in Figure~\ref{fig:flowMen} and in MSM in Figure~\ref{fig:flowQueer}.  
\begin{figure}[h!]
\begin{center}
\resizebox{0.7\textwidth}{!}{
\begin{tikzpicture}
%compartments
\node (SM)[bigcompartment, bottom color=blue!30] {{$S_M$}};
\node (SIM) [right=of SM,xshift=3cm]{{}};
\node (SVM) [above=of SM]{};

\node (IM) [bigcompartment,right=of SIM,bottom color=red!30]{{$I_M$}};
\node (VM) [bigcompartment,above=of SVM,bottom color=green!30]{{$V_M$}};
\node (leftSM) [left=of SM]{{}};
\node (leftVM) [left=of VM]{};

%arrows
\draw[very thick, ->] (leftSM) node[left]{$(1-v_m)b_mN_M$} to  (SM);
\draw[very thick, ->] (leftVM) node[left]{$v_mb_mN_M$} to  (VM);

\draw[very thick,->] (VM) to node[left]{$w_m$} (SM);

\draw[very thick,->] (SM) to node[above]{$(1-q)\frac{\beta_{fm}I_W}{N_W}$} (IM);
\draw[very thick,->] (VM) to[bend left=35] node[above,yshift=3ex]{$\eps(1-q)\frac{\beta_{fm}I_W}{N_W}$} (IM);
\draw[very thick,->] (IM) to[bend left=45] node[below] {$\gamma_m$} (SM);
\end{tikzpicture}
}% end resize box
\end{center}
\caption{HPV Transmission flow diagram for straight men.}
\label{fig:flowMen}
\end{figure}
%
\begin{figure}[h!]
\begin{center}
\resizebox{0.7\textwidth}{!}{
\begin{tikzpicture}
%compartments
\node (S)[bigcompartment, bottom color=blue!30] {{$S_Q$}};
\node (SI) [right=of S,xshift=3cm]{{}};
\node (SV) [above=of S]{};

\node (I) [bigcompartment,right=of SIM,bottom color=red!30]{{$I_Q$}};
\node (V) [bigcompartment,above=of SVM,bottom color=green!30]{{$V_Q$}};
\node (leftSQ) [left=of S]{{}};
\node (leftVQ) [left=of V]{};

%arrows
\draw[very thick, ->] (leftSQ) node[left]{$(1-v_m)b_mN_Q$} to  (S);
\draw[very thick, ->] (leftVQ) node[left]{$v_mb_mN_Q$} to  (V);

\draw[very thick,->] (V) to node[left]{$w_m$} (S);

\draw[very thick,->] (S) to node[above]{$\frac{p^2\beta_{mm}I_Q}{N_Q}+\frac{(1-p)(1-q)\beta_{mf}I_W}{N_W}$} (I);
\draw[very thick,->] (V) to[bend left=35] node[above,yshift=3ex]{$\eps(\frac{p^2\beta_{mm}I_Q}{N_Q}+\frac{(1-p)(1-q)\beta_{mf}I_W}{N_W})$} (I);
\draw[very thick,->] (I) to[bend left=45] node[below] {$\gamma_m$} (S);
\end{tikzpicture}
}% end resize box
\end{center}
\caption{HPV Transmission flow diagram for men who have sex with men.}
\label{fig:flowQueer}
\end{figure}
The diagram for women is similar to these with the appropriate transmission terms and parameters outlined above. This translates to a system of differential equations:
\begin{align}
\D{S_M}{t}&=(1-v_m)b_mN_M + w_mV_M - S_M\left(\frac{(1-q)\beta_{mf}I_W}{N_W}\right) + \gamma_mI_M - d_mS_M,\\
\D{I_M}{t}&=\left(\frac{(1-q)\beta_{mf}I_W}{N_W}\right)(S_M+\eps V_M)-\gamma_mI_M - d_mI_M,\\
\D{V_M}{t}&=v_mb_mN_M - w_mV_M - \eps V_M\left(\frac{(1-q)\beta_{mf}I_W}{N_W}\right) - d_mV_M,\\[2ex]
\D{S_Q}{t}&=(1-v_m)b_mN_Q + w_mV_Q - S_Q\left(\frac{p^2\beta_{mm}I_Q}{N_Q}+\frac{(1-p)(1-q)\beta_{mf}I_W}{N_W}\right) + \gamma_mI_Q - d_mS_Q,\\
\D{I_Q}{t}&=\left(\frac{p^2\beta_{mm}I_Q}{N_Q}+\frac{(1-p)(1-q)\beta_{mf}I_W}{N_W}\right)(S_Q+\eps V_Q)-\gamma_mI_Q - d_mI_Q,\\
\D{V_Q}{t}&=v_mb_mN_Q - w_mV_Q - \eps V_Q\left(\frac{p^2\beta_{mm}I_Q}{N_Q}+\frac{(1-p)(1-q)\beta_{mf}I_W}{N_W}\right) - d_mV_Q,\\[2ex]
\D{S_W}{t}&=(1-v_f)b_fN_W + w_fV_W - S_W\left(\frac{(1-q)\beta_{fm}I_M}{N_M}+\frac{(1-q)(1-p)\beta_{fm}I_Q}{N_Q}+\frac{q^2\beta_{ff}I_W}{N_W}\right) + \gamma_fI_W - d_fS_W,\\
\D{I_W}{t}&=\left(\frac{(1-q)\beta_{fm}I_M}{N_M}+\frac{(1-q)(1-p)\beta_{fm}I_Q}{N_Q}+\frac{q^2\beta_{ff}I_W}{N_W}\right)(S_W+\eps V_W)-\gamma_fI_W - d_fI_W,\\
\D{V_W}{t}&=v_fb_fN_W - w_fV_W - \eps V_W\left(\frac{(1-q)\beta_{fm}I_M}{N_M}+\frac{(1-q)(1-p)\beta_{fm}I_Q}{N_Q}+\frac{q^2\beta_{ff}I_W}{N_W}\right) - d_fV_W.
\end{align}

\section*{Code Pipeline}

The following information outlines the code pipeline for running these simulations.  Currently, I am using estimates from Smieja et al.'s preliminary research and altering them slightly based on relative knowledge of male-female, female-female, and male-male HPV transmission.  We will need to fit the model to data to get an accurate estimate of these transmsission parameters.  

The following is a 
\begin{figure}[h!]
\begin{center}
\resizebox{0.2\linewidth}{!}{
\begin{tikzpicture}
\node (P) [bigcompartment,bottom color=red!30]{param.R};
\node (PVF) [below=of P]{};
\node (VF) [bigcompartment,bottom color=orange!30,below=of PVF]{vectorField.R};
\node (VFS) [below=of VF]{};
\node (S) [bigcompartment,bottom color=yellow!30,below=of VFS]{sim.R};
\node (SPl) [below=of S]{};
\node (Pl) [bigcompartment,bottom color=green!30,below=of SPl]{plot.R};
\node (SPnew) [right=of S]{};
\node (Pnew) [right=of S,bigcompartment, bottom color=blue!30]{VaccParam.R};
\node (PnewSnew) [below=of Pnew]{};
\node (Snew) [bigcompartment,bottom color=green!30,below=of SPl]{.R};

\end{tikzpicture}
}% end resizebox
\end{center}
\end{figure}

\end{document}
\newpage
\bibliographystyle{plain}
\bibliography{NotesBib}

\end{document}