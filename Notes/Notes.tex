\documentclass[12pt]{article}

\usepackage{amsmath,amsthm,xspace,multirow}
\usepackage[margin=1in]{geometry}
\input{preamble.tex}
\input{sirmodellingmacros.tex}

%\newcommand{\D}[2]{\frac{\mathrm{d}#1}{\mathrm{d}#2}}
%\newcommand{\ie}{\emph{i.e., }}
%\newcommand{\eg}{\emph{e.g., }}

\usepackage{tikz}
\newdimen\mylw
\tikzset{chemeq/.style={draw,thick,double distance=2pt,onearc-onearc,/chemeq/size={#1}}}
\tikzset{chemeq/.default={.4pt 6pt}}
\pgfkeys{/chemeq/size/.code={\pgfsetarrowoptions{onearc}{#1}}}
\def\parseopts#1 #2{\xdef\myalw{#1}\xdef\myasize{#2}}
\pgfarrowsdeclare{onearc}{onearc}{%
  {\edef\x{\pgfgetarrowoptions{onearc}}\expandafter\parseopts\x}
  \mylw=\myalw
  \pgfarrowsleftextend{-\myasize-.5\mylw}
  \pgfarrowsrightextend{0pt}
}{%
  \pgfsetdash{}{0pt}
  {\edef\x{\pgfgetarrowoptions{onearc}}\expandafter\parseopts\x}
  \mylw=\pgflinewidth
  \pgfsetlinewidth{\myalw}
  \pgfpathmoveto{\pgfpoint{-\myasize}{-\myasize-.5\mylw}}%
  \pgfpatharc{180}{90}{\myasize}
  \pgfusepathqstroke
}

\title{HPV Vaccination of Boys and Men}

\begin{document}
\maketitle

\section*{Previous Work - Marek's Work}

Marek Smieja and a small research group because examining this problem during a workshop.  They set up a simple HPV transmission model in men and women.  They consider three infection status groups for both men and women: susceptible, infected, and vaccinated.  Individuals enter the model when they become sexually active (they assume age 15) at a rate $b_iN_i$, where $b_i$ is the birthrate for gender $i$ and $N_i$ is the number of individuals of gender $i$.  Individuals are separated into vaccinated or susceptible at a particular vaccination rate $v_i$ for gender $i$.  

Transmission is considered based on sexual interactions between men and women.  They consider same-sex interactions in men and women, but due to the lack of same-sex data, they set these parameters to zero when fitting the data.  The transmission of HPV in women is expressed as follows (analogously for men):
\begin{equation}
S_{W}\left(\beta_{fm}\frac{I_M}{N_M} + \beta_{ff}\frac{I_W}{N_W}\right)
\end{equation}
where $\beta_{fm}$ and $\beta_{ff}$ is the transmission of HPV onto women from men and women, respectively.  When an individuals is infected, they can clear the infection at a rate $\gamma_i$ after which the individual becomes susceptible again.  There is evidence that natural infections do not provide much protection from subsequent re-infection and does not protect significantly from infections of other HPV types.  

They also consider the vaccination isn't 100\% effective or long lasting. Vaccinated individuals' protection wanes at a rate of $w_i$ for gender $i$, after which they become fully susceptible.  Vaccinated individuals also have a protective factor of $\eps\in[0,1]$, which scales HPV transmission.  That is $\eps=0$ is full protection and $\eps=1$ is no protection.  

We adapt this model by separating the male group into two different subgroups: straight males $M$ and men who have sex with men (queer men) $Q$. In this way we can examine the effects of vaccination strategies on men and queer men.  There is a concern that the female-only vaccination strategies provide little to no herd immunity to MSM and thus they may become a reservoir for HPV. 

\section*{Adaptations to the Model}

As previously mentioned we separate the men into straight men $M$ and queer men $Q$.  Queer men have sexual interactions with both men and and women.  The proportion of same-sex interactions in men is $p$ and the proportion of same-sex interactions in women is $q$.  In this way we set up three different HPV transmission terms:

We examine the transmission of HPV in straight males.  These individuals can be infected by women solely.  
\begin{equation}
S_M\left(\frac{(1-q)\beta_{mf}I_W}{N_W}\right)
\end{equation}
Queer men can be infected by both other MSM and also women.
\begin{equation}
S_Q\left(\frac{p^2\beta_{mm}I_Q}{N_Q}+\frac{(1-p)(1-q)\beta_{mf}I_W}{N_W}\right)
\end{equation}
Women can be infected by both straight and queer men, and we also consider same sex interactions in women. 
\begin{equation}
S_W\left(\frac{(1-q)\beta_{fm}I_M}{N_M}+\frac{(1-q)(1-p)\beta_{fm}I_Q}{N_Q}+\frac{q^2\beta_{ff}I_W}{N_W}\right)
\end{equation}
We set up the entire model as a flow diagram.  The flow diagram for straight men is illustrated in Figure~\ref{fig:flowMen} and in MSM in Figure~\ref{fig:flowQueer}.  
\begin{figure}[h!]
\begin{center}
\resizebox{0.7\textwidth}{!}{
\begin{tikzpicture}
%compartments
\node (SM)[bigcompartment, bottom color=blue!30] {{$S_M$}};
\node (SIM) [right=of SM,xshift=3cm]{{}};
\node (SVM) [above=of SM]{};

\node (IM) [bigcompartment,right=of SIM,bottom color=red!30]{{$I_M$}};
\node (VM) [bigcompartment,above=of SVM,bottom color=green!30]{{$V_M$}};
\node (leftSM) [left=of SM]{{}};
\node (leftVM) [left=of VM]{};

%arrows
\draw[very thick, ->] (leftSM) node[left]{$(1-v_m)b_mN_M$} to  (SM);
\draw[very thick, ->] (leftVM) node[left]{$v_mb_mN_M$} to  (VM);

\draw[very thick,->] (VM) to node[left]{$w_m$} (SM);

\draw[very thick,->] (SM) to node[above]{$(1-q)\frac{\beta_{fm}I_W}{N_W}$} (IM);
\draw[very thick,->] (VM) to[bend left=35] node[above,yshift=3ex]{$\eps(1-q)\frac{\beta_{fm}I_W}{N_W}$} (IM);
\draw[very thick,->] (IM) to[bend left=45] node[below] {$\gamma_m$} (SM);
\end{tikzpicture}
}% end resize box
\end{center}
\caption{HPV Transmission flow diagram for straight men.}
\label{fig:flowMen}
\end{figure}
%
\begin{figure}[h!]
\begin{center}
\resizebox{0.7\textwidth}{!}{
\begin{tikzpicture}
%compartments
\node (S)[bigcompartment, bottom color=blue!30] {{$S_Q$}};
\node (SI) [right=of S,xshift=3cm]{{}};
\node (SV) [above=of S]{};

\node (I) [bigcompartment,right=of SIM,bottom color=red!30]{{$I_Q$}};
\node (V) [bigcompartment,above=of SVM,bottom color=green!30]{{$V_Q$}};
\node (leftSQ) [left=of S]{{}};
\node (leftVQ) [left=of V]{};

%arrows
\draw[very thick, ->] (leftSQ) node[left]{$(1-v_m)b_mN_Q$} to  (S);
\draw[very thick, ->] (leftVQ) node[left]{$v_mb_mN_Q$} to  (V);

\draw[very thick,->] (V) to node[left]{$w_m$} (S);

\draw[very thick,->] (S) to node[above]{$\frac{p^2\beta_{mm}I_Q}{N_Q}+\frac{(1-p)(1-q)\beta_{mf}I_W}{N_W}$} (I);
\draw[very thick,->] (V) to[bend left=35] node[above,yshift=3ex]{$\eps(\frac{p^2\beta_{mm}I_Q}{N_Q}+\frac{(1-p)(1-q)\beta_{mf}I_W}{N_W})$} (I);
\draw[very thick,->] (I) to[bend left=45] node[below] {$\gamma_m$} (S);
\end{tikzpicture}
}% end resize box
\end{center}
\caption{HPV Transmission flow diagram for men who have sex with men.}
\label{fig:flowQueer}
\end{figure}
The diagram for women is similar to these with the appropriate transmission terms and parameters outlined above. This translates to a system of differential equations:
\begin{align}
\D{S_i}{t}&=(1-v_i)d_iN_i + w_iV_I - \lambda_iS_i + \gamma_iI_i - d_iS_i,\\
\D{I_i}{t}&= \lambda_iS_i - \gamma_iI_i- d_iI_i,\\
\D{V_i}{t}&= v_id_iN_i - w_iV_i - \eps \lambda_iV_i - d_iV_i.
\end{align}
where $i$ refers to either straight men, queer men, or women.  The parameter $\lambda_i$ is the force of transmission and is a function of mixing rates.  

\subsection*{Sexual Mixing}

We consider that straight men have sex with women solely, while queer men have sex with both women and men.  The group of women contain both straight and queer women, and thus have sex with both men and women.  

We construct a transmission matrix $B$ that includes information in regards to the rate of transmission given a sexual encounter, and also the probability of sexual mixing with another member:
\begin{align}
B&=M\beta\\
&=\left(\begin{array}{c c c}
0 & 0 & (1-q) \\
0 & p^2 & (1-p)(1-q) \\
(1-q) & (1-p)(1-q) & q^2
\end{array}\right)\left(\begin{array}{c c c}
\beta_{mm} & \beta_{mm} & \beta_{mf}\\
\beta_{mm} & \beta_{mm} & \beta_{mf}\\
\beta_{fm} & \beta_{fm} & \beta_{ff}
\end{array}\right)\\
&=\left(\begin{array}{c c c}
0 & 0 & (1-q)\beta_{mf} \\
0 & p^2\beta_{mm} & (1-p)(1-q)\beta_{mf}\\
(1-q)\beta_{fm} & (1-p)(1-q)\beta_{fm} & q^2\beta_{ff} 
\end{array}\right)
\end{align}

The mixing matrix $M$ isn't quite right.  That was my first attempt at sexual mixing, where $p$ is the proportion of sexual encounters that a queer man has with another man, and $q$ is the proportion of sexual encounters that a woman has with another woman.  I have done a little bit of reading on mixing matrices, but this is all I have found.  They analyze data and the number of instances of sexual mixing.  Here is an example I found.  The rows indicate the sample participant, and the columns represent the reported partner by the sample participants.  This is a dummy example I found with just male and female mixing:
\begin{table}[h!]
\begin{center}
\begin{tabular} {c |c | c |  c c c }
 & M & F & Pairs & Persons & Contact Rate\\
 \hline
 M & 1 & 4 & 5 & 3 & 5/3=1.7\\
 \hline
 F & 4 & 1 & 5 & 2 & 5/2=2.5\\
 \hline
 & & & 10 & 5 & 2.0
\end{tabular}
\caption{Dummy example for mixing matrices just between men and women.  The sampled individual is listed on the rows and the reported partners is listed on the columns.  This was an example I got from online, so I don't know if this makes sense or is what you had in mind.}
\end{center}
\end{table}

\section*{Parameter Values}

I have taken the parameter values from the model that Marek et al. used in their project, and adapted them slightly. They set the parameter $\gamma$ based on prior knowledge about HPV clearance times, and fit the model to data to estimate $\beta$. They let vaccination parameters $v$, $w$, and $\eps$ be variable parameters in the model to examine the effects of various vaccination strategies on the prevalence of HPV in the population.

Most HPV infections are fully cleared after an average of 16 months.  Thus they set $\gamma=1/16$ month\textsuperscript{-1}.  By fitting the model to population level data in Ontario they estimate a $\beta=0.0845$ month\textsuperscript{-1}.  They don't differentiate between male-to-female or female-to-male transmission, so I assume that this is male-to-female transmission, and thus $\beta_{fm}=0.0845$ in the dummy model.  Because we consider various sexual mixing scenarios, I altered the transmission rates based on prior knowledge about HPV disease dynamics.  It has been shown that female-to-male HPV transmission is less common (apart from oropharyngeal HPV) thus I set $\beta_{mf}=0.0800$ month\textsuperscript{-1}.  Furthermore, female-to-female transmission is thought to be even less common, so I set $\beta_{ff}=0.0700$.  It is also thought that male-to-male transmission is greater than that of $\beta_{fm}$ so I set this value to be $\beta_{mm}=0.0900$ month\textsuperscript{-1}.  Because these values are anecdotal guesses, not much weight should be put into the accuracy of these values.  They purely provide a means by which to develop a skeletal model for fitting at a later date. 

In terms of sexual mixing, I have also anecdotally guess the proportion of same-sex interactions for women and queer men. 


\end{document}
\newpage
\bibliographystyle{plain}
\bibliography{NotesBib}

\end{document}